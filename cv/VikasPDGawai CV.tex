%%%%%%%%%%%%%%%%%%%%%%%%%%%%%%%%%%%%%%%%%
% Medium Length Professional CV
% LaTeX Template
% Version 2.0 (8/5/13)
%
% This template has been downloaded from:
% http://www.LaTeXTemplates.com
%
% Original author:
% Rishi Shah 
%
% Important note:
% This template requires the resume.cls file to be in the same directory as the
% .tex file. The resume.cls file provides the resume style used for structuring the
% document.
%
%%%%%%%%%%%%%%%%%%%%%%%%%%%%%%%%%%%%%%%%%

%----------------------------------------------------------------------------------------
%	PACKAGES AND OTHER DOCUMENT CONFIGURATIONS
%----------------------------------------------------------------------------------------



\documentclass{resume} % Use the custom resume.cls style

\pagestyle{plain} 
\usepackage[utf8]{inputenc}
\usepackage[english]{babel}
\usepackage{fancyhdr}
\usepackage{lastpage}
\pagestyle{fancy}
\fancyhf{}
 
\rfoot{Page \thepage \hspace{1pt} of \pageref{LastPage}}%\usepackage{lmodern}
\usepackage[left=0.57in,top=0.3in,right=0.57in,bottom=0.3in]{geometry} % Document margins
\usepackage{hyperref}
\usepackage{comment}
\hypersetup{
    colorlinks=false,
    linkcolor=black,
    filecolor=magenta,      
    urlcolor=cyan,
}
\newcommand{\tab}[1]{\hspace{.2667\textwidth}\rlap{#1}}
\newcommand{\itab}[1]{\hspace{0em}\rlap{#1}}
\providecommand*\email[1]{\href{mailto:#1}{#1}}
\name{\Large{Vikas PD Gawai}} % Your name
\address{520, Taylor Hall, 427 Lorch St, Madison, 53706, WI, USA} % Your address
%\address{123 Pleasant Lane \\ City, State 12345} % Your secondary addess (optional)
\address{Phone: 608-628-8674 \\ \email{gawai@wisc.edu} \\ \url{https://vikasgawai.github.io/home/}} % Your phone number and email

\usepackage{multicol}

\newenvironment{innerlist}[1][\enskip\textbullet]%
        {\begin{itemize}[#1,leftmargin=*,parsep=0pt,itemsep=0pt,topsep=0pt,partopsep=0pt]}
        {\end{itemize}}



\begin{document}

%----------------------------------------------------------------------------------------
%	EDUCATION SECTION
%----------------------------------------------------------------------------------------

\begin{rSection}{Education}

{\bf University of Wisconsin-Madison}, Ph.D., Ag and Applied Economics (AAE) \hfill {\em Expected 2024} 
%\\ \qquad Dissertation Title: \textit{Essays on Public Policy and Household Insecurities}
%\\ Committee: Paul Dower (chair), Bradford Barham, Laura Schechter, Jeremy Foltz, and Mariel Barnes
%\\ M.Sc., Agricultural and Applied Economics \hfill {\em 2019}


{\bf University of Wisconsin-Madison}, Masters of Economics \hfill {\em 2016-18}
%Advanced Study: Smittcamp Family Honors College, Craig Business Honors Program
%Member of Eta Kappa Nu \\
%Member of Upsilon Pi Epsilon \\

{\bf University of Mumbai, India}, Bachelor of Technology, Civil Engineering \hfill {\em 2004-08}

\end{rSection}

%\begin{rSection}{Research Interests}
% \textit{Primary}: Public policy, public health, and applied microeconomics. \\
% \textit{Secondary}: Environmental health, climate change, and economics of conflict and violence.
%\end{rSection}

%\begin{rSection}{Academic Positions}
%	{\bf University of Denver, Josef Korbel School of International Studies} \\
%	Assistant Professor of Microeconomics for Public Policy \hfill \textit{2022--}
	
%	{\bf Pontificia Universidad Cat\'{o}lica de Chile} \\
%	{Visiting Scholar, Instituto de Estudios Urbanos }{\hfill \em 2018 }
%\end{rSection}
%--------------------------------------------------------------------------------
%    Projects
%-----------------------------------------------------------------------------------------------
%\begin{rSection}{Refereed Publications}
%\textbf{Sims, K.M.}, Foltz, J.D., and M.E. Skidmore. (2021) ``Prisons and COVID-19 spread in the United States.'' \textit{American Journal of Public Health}, 111(8): 1534-1541. %\hyperref{https://doi.org/10.2105/AJPH.2021.306352.}{https://doi.org/10.2105/AJPH.2021.306352.}

%\end{rSection}

\begin{rSection}{Research Interests}
Health, Aging, Social Security, Public Policy, Human Capital, Caste (Race), Gender, Agriculture, Broadband
\end{rSection}


\begin{rSection}{Job Market Paper}
\begin{itemize}
\item Does High-Speed Internet (Broadband) Technology Affect the Mental Health of Older Adults? \textit{\scriptsize{(Presentations - MEA '23, American University '23, SEA '23, APPAM '23); Media Highlight: \href{https://grow.cals.wisc.edu/departments/features/the-road-to-rural-broadband}{GROW MAGAZINE cover story}}} 
\end{itemize}

\\
\textbf{Abstract-} 
Recent research has shown the negative effects of social networks on younger people's mental health. Yet, we do not know whether the same effects are present across the population, particularly for older adults, who are especially vulnerable to certain mental health conditions.
I estimate the effect of broadband availability on the mental health of older adults (aged 50+) in the United States, using individual panel data and exploiting the census tract-level staggered rollout of broadband from 2010 to 2018 in the latest difference-in-differences (DID) framework.
In contrast to the literature that finds harmful effects of the internet among younger populations, my results show that broadband rollout significantly reduces depression symptoms by 5.7\% among older adults. The primary mechanisms driving these positive effects include an increase in self-use of the internet and virtual social connectedness and a decline in social isolation and loneliness. Improved health literacy and technological efficiency (telehealth) also partly drive the results but do not affect self-reported physical health or employment.
Rural dwellers and women are the biggest beneficiaries of broadband's positive effects on mental health.
I also find some evidence of the spatial spillover effects.
These positive estimates are comparable with other major life events known to negatively affect the mental health of older adults, such as job loss, recession, and the death of a spouse. This work highlights broadband's unmeasured additional benefits to public investments, given the recent allocation of over $\$$65 billion to broadband expansion. A back-of-the-envelope calculation suggests that broadband expansion may reduce the cost of excess Medicare spending by about $\$$5 billion due to major depressive symptoms and social isolation among older adults.  
%The paper further uncovers other underexplored mechanisms, including health literacy, cognitive function, and technological efficiency (telehealth) in nearby hospitals. With recent massive public investments of over \$65 Billion in broadband infrastructure, these results carry significant policy implications for broadband policies and emphasize the potential benefits for older adults.
%Heterogeneous effects by race, gender, and geography highlight higher benefits for women and rural areas while emphasizing potential barriers for non-Whites.
%benefit more significantly from broadband, suggesting the potential of broadband to reduce the gender and spatial gap in mental health .
%I find an inverted U shape of the positive effects concentrated among the 65 to 85 age groups.
%I find evidence of novel mechanisms like an increase in \textit{social connectedness} and a decline in \textit{social isolation}.
%These contrasting findings for young and old cohorts underscore how similar technologies can affect differently by age and behavior.
%Recent empirical evidence in economics underscores the adverse impact of internet-enabled technology (social media) on college students' mental health, mainly due to unfavorable \textit{social comparisons}.
%Leveraging the quasi-experimental staggered rollout of high-speed internet broadband,
%Yet, it remains unclear how a comparable technology (high-speed internet broadband) affects a more vulnerable older population.


\end{rSection}

\begin{rSection}{Publications}
\begin{itemize}


\item Early-Life Economic Conditions and Old-Age Mortality: Evidence from Historical County-Level Bank Deposit Data, \textit{with Valentina Duque, Jason Fletcher, Hamid Noghanibehambari, Lauren Schmitz}-  \textbf{(Accepted at \textit{Journal of Population Economics, January 2024)}}
\iffalse \textbf{Abstract-}
This paper studies the long-run mortality effects of in-utero and early-life economic conditions. We examine
how local economic conditions experienced in the Great Depression, proxied by county-level banking
deposits during in-utero and first years of life, can influence old-age longevity. We find that a one-standard-deviation rise in per capita bank deposits is associated with an approximately 2.8 months increase in
longevity at old ages (a 0.4 percent increase with respect to the outcome mean). The effects are robust across
a wide array of specification checks. Additional analyses comparing state-level versus county-level
economic measures provide insight into the importance of controlling for local-level confounders and
exploiting more granular measures in exploring the relationship between early-life conditions and later-life
mortality.
\fi


\item Discrimination in Science: Salaries of Foreign and US Born Land-Grant University Scientists  \textit{with Jeremy Foltz}
 \textbf{(R\&R at \textit{American Journal of Agricultural Economics (AJAE)})} %\textit{\scriptsize{%(Presentation - AAEA '22)}}
\iffalse\textbf{Abstract-} 
The dominance of the US innovation and academic system relies heavily on foreign-born labor for its success.
Recent literature has shown evidence of wage gaps in academia based on gender and race; however, little is known about whether a wage gap might exist for foreign-born faculty.
This paper studies the wage gap between the US and foreign-born agricultural and life science faculty at 52 US Land Grant Universities (LGU) using a survey of over 1,400 scientists conducted in 2005 and 2015.
We develop a framework to categorize the sources of a potential wage gap into testable categories that capture \textit{direct} discrimination as well as indirect (\textit{systemic}) discrimination.
We find that among the tenure-track faculty, foreign-born earn about 4\% or \$5,200 lower annual wages even though, on average, foreign-born scientists work more hours per week and produce about 52\% more journal articles than US-born scientists. The estimated wage gap is robust to a range of alternative empirical specifications.
The decomposition analysis suggests that about one-third of the wage gap is due to direct discrimination, and about two-thirds is due to various types of systemic discrimination.
Using our framework, we then rule in and rule out some important types of systemic discrimination.
Estimates from this paper are crucial for understanding potential policies that could improve diversity, equity, and inclusion in US academia.
\fi

\end{itemize}


\begin{rSection}{Working Papers}
\begin{itemize}
\item Effects of Early Life Exposures to the Green Revolution (agricultural technology) on Later Life Aging Outcomes: Evidence from the Largest Aging Data \textit{with Lauren Schmitz, Valentina Duque, Jinkook Lee, Hunter Green} (\textit{\scriptsize{presented at APPAM '22}}) 

\item Effects of Broadband on Social Security Insurance (SSI) and Disability Insurance (SSDI) Applications \textit{\scriptsize{(Award from Retirement and Disability Research Center (\textbf{RDRC}) Center for Financial Security (CFS), Social Security Administration (\textbf{SSA}))}}

\item Spillover Effects of a Bicycle Policy on Girls' Enrollment in Middle Schools \textit{\scriptsize{(Recognition under Best Paper Scholarship (Honorable mention), Dept. of Economics, UW-Madison, 2018) }}


\end{itemize}




%{\bf Productive cattle ranches reduce carbon emissions in the Brazilian Amazon} 
%\textit{with M.E. Skidmore, L. Rausch, and H.K. Gibbs \textit{(under review)}.}   

%{\bf Turning a house into a home: Delayed property rights and education investment decisions in urban Chile} \textit{(under review)}.

%{\bf Seeking safe harbors: Emergency domestic violence shelters and family violence} \textit{(job market paper).}


\end{rSection}


\begin{rSection}{Selected Work in Progress}

\item Is there a Link between Access to Broadband and Health Outcomes? A spatial econometrics approach using County Health Rankings data \textit{with Steven Deller}

\item Spatially Varying Gender and Caste Inequalities in Cognition and the Contribution of Individual and State-level Characteristics Across India \textit{with Lauren Schmitz, Kanika Arora, and Shubhashrita Basu} \textit{\scriptsize{(selected at a Workshop on Determinants of Adult Mortality, Morbidity, and Healthy Aging in LMICs at UC, Berkeley)}}

\item Effects of Teacher Hiring Policy on Student Test Scores in Rural India: A Multiple Regression Discontinuity Design Approach for one of the Largest Public School Systems in the World \textit{\scriptsize{(selected at American Education Finance and Policy conference (AEFP))}}


\item Effects of Rural Hospital Closures on Mortality in Severe Weather Conditions \textit{\scriptsize{(selected at Association of Environmental and Resource Economists (AERE))}}

\item Effect of Green Revolution on the Intergenerational Human Capital \textit{with Valentina Duque, Lauren Schmitz} \textit{\scriptsize{(presented at PAA '22})}
%-Effect of Food Stamp Programs on Occupational Mobility in the US



\end{rSection}


%----------------------------------------------------------------------------------------
%	TEACHING EXPERIENCE SECTION
%----------------------------------------------------------------------------------------


\begin{rSection}{Competitive Honors}
-\textbf{\textit{Diversity Fellow}}, School of Public Affairs, American University \hfill \textit{2023-} \\
-\textbf{\textit{Research Fellow}}, Retirement and Disability Research Center (\textbf{RDRC}) Center for Financial Security (\textbf{CFS}), Social Security Administration (\textbf{SSA})  \hfill \textit{2022-} \\
-\textbf{\textit{DEI- Diversity Scholarship}} American Society of Health Economics (ASHEcon) \hfill \textit{2021} \\
-\textbf{\textit{Best Research Paper Scholarship (Honorable mention)}} Dept. of Economics, UW Madison \hfill \textit{2018}
 %UW-Madison Sexual Violence Research Initiative%, Research Affiliate 
%\hfill \textit{2020--}\\
%Gibbs Land Use and the Environment Lab%, Faculty Affiliate 
%\hfill \textit{2020--}
\end{rSection}



%---------------------------------------------------------------------------------------- Awards%----------------------------------------------------------------------------------------

\begin{rSection}{Competitive Grants and Awards}
Center for Demography of Health and Aging \textbf{Research Travel} Award- \$2,000 \hfill \textit{2023}	\\
RDRC - Junior Scholar \textbf{Research Competition} Award- \$5,000 \hfill \textit{2022-23}	\\
Prof. Jeremy Foltz \textbf{Research Travel} Award- \$1,100 \hfill \textit{2023}	\\
Deborah and David Penn Fund - \textbf{Research Presentation} Award- \$500, \$1000 \hfill \textit{2022, 23}	\\
UW-Madison- Graduate School - \textbf{Research Presentation} Award- \$2,500, \$1,500 \hfill \textit{2022, 23}	\\
UW-Madison- CDE \textbf{Research Presentation} Award- \$950 \hfill \textit{2022} \\
UW-Madison- Traisman Agribusiness \textbf{Graduate Fellowship}- \$750 \hfill \textit{2021}	\\
UW-Madison Student Research Grants Competition \textbf{Research Travel}- \$1,500 \hfill \textit{2020}  \\
UW Madison Graduate Student \textbf{Summer Fieldwork} Award- \$3,000  \hfill \textit{2019}\\
Indian State Government Scholarship for Masters in Economics- $~$ \$100,000 \hfill \textit{2016-18}

\end{rSection}

%	conferences
%----------------------------------------------------------------------------------------
\begin{rSection}{Academic Affiliations}
-Graduate Student, Center for Demography of Health and Aging (\textbf{CDHA}), and Center for Demography and Ecology(\textbf{CDE})
 \hfill \textit{2019--Current}\\
-Graduate Student, Institute of Research on Poverty (\textbf{IRP})%, Faculty Affiliate 
 \hfill \textit{2021-2022}\\
 -Graduate Student, Center for South Asia, (\textbf{CSA})%, Faculty Affiliate 
 \hfill \textit{2020--}
%UW-Madison Sexual Violence Research Initiative%, Research Affiliate 
%\hfill \textit{2020--}\\
%Gibbs Land Use and the Environment Lab%, Faculty Affiliate 
%\hfill \textit{2020--}
\end{rSection}




\begin{rSection}{Conference Presentations} 
	Midwest Economics Association (MEA) {\hfill \em 2023}\\
	Agricultural and Applied Economics Association (AAEA) {\hfill \em 2022}\\
	Association for Public Policy Analysis \& Management (APPAM) {\hfill \em 2022, 23} \\
	Population Association of America (PAA) {\hfill \em 2022} \\
	Southern Economics Association (SEA) {\hfill \em 2023} 





	
	
%	\textit{2022}: Association for Public Policy Analysis \& Management
	
%	\textit{2021}:	Workshop on the Economics of Crime for Junior scholars; University of Georgia Agricultural \& Applied Economics Department Seminar; Southern Economics Association {(session canceled)}; Agricultural and Applied Economics Association; Western Economic Association International’s Graduate Student Workshop; University of Chicago Crime Lab; Midwest Economics Association; Association for Mentoring In Economics
	
%	\textit{2019}: Ronald Coase Institute
	
%	\textit{2016}: 	Western Social Science Association
	
%	\textit{2015}: California State University Honors Conference, Plenary Speaker; SSRIC Social Science Student Symposium
\end{rSection}
	
%%%%%%% SERVICE 
\begin{rSection}{Leadership, Mentorship}%, Memberships, and Workshops }
%\textit{Referee For:} Land Economics \\ 
%\textit{Department Service:} 
\textit{Mentor for two undergrad students on a class project under Prof. Foltz} {\hfill\textit{2022}}\\ 
\textit{Mentor for three FirstGen undergrad student} {\hfill\textit{2022}}\\
UW-Madison First Generation (\textbf{\textit{FirstGen}}), \textit{Member} {\hfill\textit{ 2021--}}\\
UW-Madison CHDA NextGen Population, \textit{Mentor for 18 students on research development} {\hfill\textit{2022}}\\
UW-Madison, AAE, Faculty Hiring Committee, \textit{Student Representative} {\hfill \textit{2022}}\\
UW-Madison, AAE, Ph.D. Admission Committee, \textit{Student Representative} {\hfill \textit{2021}}\\
Taylor-Hibbard Club, \textit{Athletic Chair} {\hfill \textit{2020--2022}} \\
UW-Madison AAE, Development Lab Group, \textit{Member} {\hfill \textit{2019--}}
\end{rSection}



\begin{rSection}{Academic Service}%, Memberships, and Workshops }
%\textit{Referee For:} Land Economics \\ 
%\textit{Department Service:} 
Association for Public Policy Analysis \& Management (APPAM),  \textit{Abstract Reviewer}  {\hfill \em 2023} \\
Journal- Economics of Education Review, \textit{Reviewer} {\hfill\textit{2022-23}} \\ 
American Society for Health Economists (AshEcon), \textit{Abstract Reviewer} {\hfill\textit{2022}} \\ 
Agriculture and Applied Economics (AAEA), \textit{Abstract Reviewer} {\hfill \textit{2022, 2023}}\\
AAEA-Health Economics Section (Chair- Prof. Di Fang) \textit{Official Twitter Manager (with Lei Pan)} {\hfill \textit{2022-23}}  
\end{rSection}



\begin{rSection}{Research Experience}
%{\bf Ideas42}	\\
%Consultant {\hfill \em 2021--}
\textbf{Research Assistant} to Prof. Lauren Schmitz (UW Madison, School of Public Affairs) and Prof. Valentina Duque (University of Sydney, Princeton University, American University) \hfill \textit{2020--} \\
\textbf{Research Assistant} to Prof. Jeremy Foltz and Prof. Bradford Barham (UW Madison, AAE) \hfill \textit{2018-2020} \\
\textbf{Research Assistant} to Prof. Andra Ghent (UW Madison, Wisconsin School of Business) \hfill \textit{May-Sept (2018)}
\end{rSection}


\begin{rSection}{Teaching Experience}
%{\bf Ideas42}	\\
%Consultant {\hfill \em 2021--}
Guest Lecturer: \textbf{Agricultural \& Economic Development of Africa (Health)}, (Instructor- Osaretin Olurotimi) UW-Madison, AAE 477
 \hfill \textit{Spring 2022} \\
Guest Lecturer: \textbf{International Studies (Gender Gap and Economics Development in Developing Countries)}, (Instructor- Prof. Jeremy Foltz) UW-Madison, AAE 374
 \hfill \textit{Spring 2022, 2023} \\
Guest Lecturer: \textbf{The Economics of Race and Gender} (Gender Gap in Developing Countries), (Instructor- Prof. Meghan I. Esson) Bentley University, EC 298
 \hfill \textit{Spring 2022}
\end{rSection}



\begin{rSection}{Technical Skills}
STATA, R(basic), Python(basic), Latex



%%%% REFERENCES
\begin{rSection}{References}
	\begin{multicols}{2}
\textbf{Jeremy Foltz (advisor)} \\
Professor, Agricultural and Applied Economics \\ 
University of Wisconsin-Madison \\
 {\email{jdfoltz@wisc.edu}}, {}

\textbf{Lauren Schmitz} \\
Assistant Professor, La Follette School of Public Affairs \\
University of Wisconsin-Madison \\
 {\email{llschmitz@wisc.edu}}, {}

%\textbf{Valentina Duque}\\
%Assistant Professor, Department of Public Administration and Policy \\
%American University \\
%\hfill{\email{vduque@american.edu}}

\textbf{John Mullahy}\\
Professor, Population Health Sciences \\
 University of Wisconsin-Madison\\
{\email{jmullahy@wisc.edu}}, {} 


\textbf{Priya Mukherjee}\\
Assistant Professor, Agricultural and Applied Economics \\
 University of Wisconsin-Madison\\
{\email{priya.mukherjee@wisc.edu}}, {} 


%\textbf{Paul Casta\~neda Dower}\\
%Associate Professor, Agricultural and Applied Economics \\
% University of Wisconsin-Madison\\
%{\email{pdower@wisc.edu}}, {(608) 262-4499}
\end{multicols}


%\textbf{Kate Walsh}\\
%Associate Professor, Psychology and Gender \& Women's Studies \hfill \\
%University of Wisconsin-Madison
%\hfill{\email{klwalsh2@wisc.edu}}

\end{rSection}



%\(^*\) indicates the event/course was virtual. 

\textit{Updated \today}




\end{document}